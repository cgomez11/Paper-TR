% All papers should start with an Introduction section, which sets the work
% in context, cites relevant earlier studies in the field by \citet{Others2013},
% and describes the problem the authors aim to solve \citep[e.g.][]{Author2012}.

%%% NEW MULTI EPOCH SURVEYS ENABLE STUDY AND DETECTION IN CRAZY SCALES
The study and detection of astronomical variable sources is expected
to occur on unprecedented scales with the new generation of
forthcoming multi-epoch and multi-band (synoptic) astronomical
surveys. 
For instance, the Large Synoptic Survey Telescope (LSST)
\citep{0805.2366}, one of the largest synoptic survey telescopes to
come in the following years, will generate approximately 15 terabytes
of data every night \citep{1512.07914}.  
%Such telescope is expected to detect and alert about 10 million possible transients too. 
Other surveys including the Square Kilometer Array (SKA) are also expected
to generate exuberant daily data-streams. 

%%% HISTORICALLY IT HAS BEEN DONE MANUALLY, NOW IT ISN"T POSSIBLE
This observational leap renders manual classification techniques
unfeasible.   
Traditionally, such objects have been classified by visual inspection
by experts or through crowd-sourcing
\citep{1011.2199,0708.2750}. 
This is approach is slow and expensive.
Another concern is the possible biases and difficulty to standardize among 
astronomers \citep{1104.3142}. 
Alternatively, transient detection can be executed much faster than
human astronomers through computational techniques using machine
learning, 
which are deterministic and calculate the results' degrees of
certainty. 
These methods also allow for real-time triggering of follow-up
observations that optimize the economical and temporal resources. 
 

%%% TRANSIENTS EVENTS ARE CHALLENGES (AND WHAT TRANSIENT EVENTS ARE)
Another challenge in  Time Domain Astronomy is Real-Time Transient classification. 
Astronomical Transients are events which's luminosity varies in short duration
relative in the timescale of the universe, from minutes to several
years. 
Transients include phenomena such as supernovae, novae, neutron
stars, blazars, pulsars, cataclysmic variable stars (CV), gamma ray
bursts (GRB) and active galaxy nucleus (AGN). 
The time-domain dependency of these objects is one of the reasons why
they are hard to classify: their data is usually heterogeneous,
unbalanced, sparse, unevenly sampled and with missing information. 
Automating the recognition and classification of transient events, a type of such
variable sources, would reduce costs and speed up this process as well
as provide scientists information of the universe in various spatial
scales \citep{2011arXiv1110.4655D}.  

%%% HOW DOES TRANSIENT DETECTION
Transient detection is generally done through 
difference imaging \citep{1507.05137,1608.01733,1708.02850}. 
This algorithm starts by aligning the image of interest on a reference image of the
same region of sky, processing the former to match its point-spread
function in all regions with the latter's, and subtracting both
\citep{astro-ph/9712287}. 
As a result, variable stars and transient objects which were not
visible in the reference image will remain in the resulting
image. 
Nonetheless, the difference images can also contain bogus
artifacts due to imperfections in the image processing
phase, in the telescope or other phenomena. 
Unfortunately, distinguishing between bogus and real transient objects
still requires human intervetion, making it expensive and slow
\citep{2011arXiv1110.4655D}.  


%%% STATE OF THE ART I: REAL VS BOGUS
There are succesful attempts to implement automatic detection
algorithms and distinguish bogus artifacts from real transient
objects.  
For instance, raw images from the  Skymapper Supernova and Transient
Survey and the High cadence Transient Survey (HiTS) have been used as
inputs for automatic detection algorithms \citep{1708.08947,1701.00458}.
Convolutional Neural Networks (CNN) have also achieved
high accuracy in this binary classification task.
Other studies have shown that bogus artifacts can be detected using
features extracted from raw images. 
\cite{1601.06151} and \cite{1601.06320}, achieved reliable
classification by transforming transient data from the OGLE-IV
data-reduction pipeline and training it with machine learning
algorithms such as Artificial Neural Networks, Support Vector
Machines, Random Forests, Naive Bayes, K-Nearest Neighbors and Linear
Discriminant Analysis.  
Similarly, \cite{1501.05470} used images from Pan-STARS1 Medium Deep
Surveys, and \cite{1407.4118} processed single-epoch multi-band images
from the SDSS supernova survey for the same purpose.  


%%% STATE OF THE ART II: DIRECT

Alternative approaches try to classify astronomical transient events directly, skipping the difference imaging process.
This is usually done by extracting relevant features (e.g. periodic, non-periodic) from the astronomical object light-curves, and using machine learning algorithms for classification. For instance, in \cite{1401.3211} and \cite{1601.03931}, light curves of objects six or more transient classes from the Catalina Real Time Transient Survey and the Downes set (\cite{d05}) were classified. Other studies which research a similar approach include \cite{1603.00882}, where supernovas from the Supernova Photometric Classification Challenge were automatically classified.

Direct use of light curves as data inputs is also currently under research. Recurrent Neural Networks (RNNs) have been proven succesful to classify transient events. In \cite{1606.07442} and \cite{1710.06804} it is evident how more modern algorithms can also learn from time-series data without much pre-processing.

% PROPOSAL
In this work we test different techniques to automatically classify transient events with machine learning, using their light curves. Namely: filtering them by their observation count, oversampling unbalanced light curve classes and extracting several statistical descriptors from the resulting data. Such generated statistics are used as input to three different machine learning models with several hyper-parameter variations: Support Vector Machines, Neural Networks and Random Forests, which automatically learn to detect and classify between different types of transient objects. With extensive testing, multiple experiments were executed in order to find the best training parameters. This lead to state of the art and novel experimental results.

% STRUCTURE
The paper is structured as it follows: in Section \ref{section_data} we present the dataset used for this project. Section \ref{section_method} describes the methodology proposed. Section \ref{section_experimentation} explains the experiments performed to classify transient objects with machine learning. Finally, the results are discussed in Section \ref{section_results}.

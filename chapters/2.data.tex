%%% Overview

In this work we make use of data from the Catalina Real-Time Transient Survey (CRTS) (\cite{1111.2566}). As it's name implies, the CRTS is an astronomical survey in the search of transient and highly variable objects. It covers 33,000 squared degrees of the sky in search of this kind of objects and has surveyed astronomical sources since 2007. Three telescopes are used to capture data from the sky: Mt. Lemmon Survey (MLS), Catalina Sky Survey (CSS), and Siding Spring Survey (SSS). So far, CRTS has discovered more than 15.000 transient events.

Specifically, this work uses the light curves of 4869 unique transient events that contain at least 5 observations, which were detected with the CSS telescope of the CRTS. This f/1.8 Schmidt telescope is located in the Santa Catalina Mountains, north of Tucson, Arizona. It is equipped with a 111-megapixel (10,560 x 10,560 pixel) detector, and covers 4000 square degrees per night, with a limiting magnitude of 19.5 in the visual filter band
.%\cite{css}.

All transient objects are classified in the CRTS data-set according to their type. The most relevant classes found are: supernovae (SN), cataclysmic variable stars (CV), blazars, flares, asteroids, active galactic nuclei (AGN), and high-proper-motion stars (HPM). Though most objects in the transient object catalogue belong to a single class, there's uncertainty is present in the categorization of some of them. (an interrogation sign is used when a class is not clear e.g. SN? or sometimes multiple possible classes are found for a single event e.g. SN/CV).

Furthermore, the data-set used in this project contains information of 16940 non-transient sources. Sources in this data-set were selected by sampling light curves of objects within a 0.006 degree radius from CRTS detected transients, and removing known transient light curves from this set. This process should return only non-transient sources, though it is possible that non-detected transients were captured and incorrectly catalogued as 'non-transients'.

Note that when capturing information with telescopes, errors may arise in the measurements. This is due to multiple factors, such as the variable atmospheric conditions affecting the capturing device. Both the transient and non-transient light-curve catalogues contain photometric errors for each observation in every light-curve. In the CRTS, these uncertainties have been calculated with an empirical relationship between the source flux and the observed photometric scatter .%\cite{CRTS_FAQ}. 
Such relation was derived based on 100,000 isotropically selected sources that showed non-significant variability based on their Welch-Stetson index (\cite{1996PASP..108..851S}).

%%% FILTERED DS METRICS....


%%% Overview

We use public data from the Catalina Real-Time Transient Survey
(CRTS) \citep{1111.2566}, an astronomical survey searching transient
and highly variable objects.   
It covered 33000 squared degrees of sky and took data since 2007.
Three telescopes were used: Mt. Lemmon Survey (MLS), Catalina Sky
Survey (CSS), and Siding Spring Survey (SSS). So far, CRTS has
discovered more than $15000$ transient events.
We use light curves as measured with the CSS telescope of the CRTS, which is
an f/1.8 Schmidt telescope located in the Santa Catalina Mountains, north of Tucson,
Arizona and is equipped with a 111-megapixel  detector, and covered
4000 square degrees per night, with a limiting magnitude of 19.5 in
the V band.  
The public CRTS data base reports the source flux and its
corresponding uncertainty \citep{1996PASP..108..851S}.


All transient objects were classified in the CRTS data-set according to
their type. 
The most relevant classes found are: supernovae (SN),
cataclysmic variable stars (CV), blazars, flares, asteroids, active
galactic nuclei (AGN), and high-proper-motion stars (HPM). 
Though most objects in the transient object catalogue belong to a single class,
there is some uncertainty in the categorization of some of
them. 
In this case, an interrogation sign is used when a class is not clear
e.g. SN? or sometimes multiple possible classes are found for a single
event e.g. SN/CV)
Table \ref{Top-Transient-Classes} summarized The number of objects in each class.


We use the light curves of $4269$ unique transient events that
contain at least 5 observations.
We also use  $15193$ non-transient sources with at least 5 observations each. 
Sources in this data-set were selected by sampling light curves
of objects within a 0.006 degree radius from CRTS detected transients,
and removing known transient light curves from that set. 
Though this process should return only non-transient sources, it is
possible that non-detected transients were captured and catalogued
incorrectly as 'non-transients'.  
Table \ref{Transient-Observation-Count} and
Table \ref{Non-Transient-Observation-Count} summarize some statitics
on the number of observations available for each light curve for the
transient and non-transient data-sets, respectively.  



\import{./figures/ObjCount/}{5.tex}


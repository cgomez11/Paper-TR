% This is a simple template for authors to write new MNRAS papers.
% The abstract should briefly describe the aims, methods, and main results of the paper.
% It should be a single paragraph not more than 250 words (200 words for Letters).
% No references should appear in the abstract.



The arrival of massive multi-epoch and multi-band astronomical surveys
demands the development of computational techniques to automate the
study and detection of transient astronomical sources. 
In this work we characterize various machine learning algorithms to
recognize and classify such events.
We use light-curves from the Catalina Real Time Transient Survey
(CRTS) to classifity thousands of unique transient and non-transient
event light-curves.
we experimented  with multiple data pre-processing methods,
feature selection techniques and self-learning models. State of the
art results were obtained in five classification tasks, particularly
there was one binary and four multi-class classification tasks. 
In our research we discovered that the best performing algorithm was Random Forests, which achieved a f1-score of 87.27\% on binary (transient \& non-transient) classification. Six-class transient classification achieved a 77.54\% f1-score, and an additional 66.39\% f1-score when including an ambiguous sources class with new samples. Variations on the multi-class transient classifications mentioned above, those which included non-transient sources, resulted in a 75.05\% f1-score and a 66.05\% f1-score respectively.

% This is a simple template for authors to write new MNRAS papers.
% The abstract should briefly describe the aims, methods, and main results of the paper.
% It should be a single paragraph not more than 250 words (200 words for Letters).
% No references should appear in the abstract.



The arrival of massive multi-epoch and multi-band astronomical surveys
demands the development of computational techniques to automate the
study and detection of transient astronomical sources. 
In this work we characterize various machine learning algorithms to
recognize and classify such events.
We use light-curves from the Catalina Real Time Transient Survey
(CRTS) to classifity thousands of unique transient and non-transient
event light-curves.
We experiment  with multiple data pre-processing methods,
feature selection techniques and self-learning models. 
We obtain new results in five classification tasks.
We pay particular attention to a binary and a four multi-class
classifications task. 
We conclude that the best performing algorithm was
a Random Forest with an f1-score of 87.27\% on binary
(transient \& non-transient) classification. 
Six-class transient classification achieved a 77.54\% f1-score, and a
66.39\% f1-score when including an ambiguous sources class.
Multi-class transient classifications including non-transient sources,
resulted in a 75.05\% f1-score and a 66.05\% f1-score respectively.  

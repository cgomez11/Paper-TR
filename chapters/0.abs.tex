% This is a simple template for authors to write new MNRAS papers.
% The abstract should briefly describe the aims, methods, and main results of the paper.
% It should be a single paragraph not more than 250 words (200 words for Letters).
% No references should appear in the abstract.


With the upcoming arrival of new generation multi-epoch and multi-band astronomical surveys, there is a demand for computational techniques that automate the study and detection of astronomical sources. One such kind of sources are transient events, which pose a difficulty in recognition that no other astronomical object have, due to their time-dependant nature. In this work we propose a new process to recognize and classify such kinds of events using machine learning algorithms. Using thousands of unique transient and non-transient event light-curves from the Catalina Real Time Transient Survey (CRTS), we experimented  with multiple data pre-processing methods, feature selection techniques and self-learning models. State of the art results were obtained in five classification tasks, particularly there was one binary and four multi-class classification tasks. In our research we discovered that the best performing algorithm was Random Forests, \textit{which scored a recall of 89.39\%} on binary (transient \& non-transient) classification. Six-class transient classification scored a \textit{77.86\% recall}, and an additional \textit{66.91\% recall} when including an ambiguous sources class with new samples. Variations on the multi-class transient classifications mentioned above, those chich included non-transient sources, resulted in a \textit{77.31\% recall} and a \textit{68.23\% recall} respectively. 
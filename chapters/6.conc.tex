The scope of forthcoming of large astronomical synaptic surveys such
as the LSST \citep{0805.2366} motivates the development and
exploration of authomatized ways to detect transient sources.
In this paper we presented an approach for the automatic recognition of transient
events with machine learning techniques. 

The method we present is based on the study of light curves. 
We extracted its characteristic features to use them as inputs
to train three different machine learning algorithms: Random Forests,
Neural Networks and Support Vector Machines.
The features extracted from light curves were either statistical
descriptors of the observ ations, or polynomial curve fitting
coefficients applied to the light curves.   

The machine learning algorithms performed a variety of classification tasks.
Namely, binary classification among transients and non-transients, and
multi-class classification of various transient classes, sometimes
including non-transients too. Detailed description of these tasks can
be found in Section \ref{section_method}.  
Overall, the best classifier for all tasks was the Random Forest,
followed by Neural Networks and then Support Vector Machines. 

State of the art results were obtained when testing the trained models
to unseen sets of data, specifically in binary and transient
multi-class classification. 
Recall scores obtained from the best classifier for each task, are
comparable to those results found in \cite{1401.3211}
and \cite{1601.03931}.   


We also investigated the parameters that produced the best results.
Training with light curves which contained 10 observations minimum
generally outperformed those with minimum 5. 
Overall, the best results were also achieved using all 30 features per
light curve, while using 20 features performed worst. 
Such phenomena may be explained due to the lower noise that curves
with more observations provide.  
Since light-curves with higher observation counts may be better
defined, they could generate a more precise higher rank polynomial
fitting than non-transients, which were the newly proposed features
found in the 26 and 30 feature sets. 

In general, training with unbalanced-class data sets resulted in the
highest values. 
Contrary to what was expected, the usage of oversampled light-curves
may be a factor that biases the model during training. 
This could be explained if such artificial light curves were too
similar to the original ones, and thus the algorithms over-fitted
during training.  

We studied feature importance using Random Forests. 
The most importante feature was always stetson\textunderscore j, followed by:
amplitude, standard deviation, skewness and median absolute
deviation. 
Conversely, 4th level polynomial fitting coefficients were
the ones which provided the least relevance, though using them was
still better, since the highest scores were a result of using all 30
features. 
Furthermore, lower polynomial fitting features like
poly2\textunderscore t1 and poly1\textunderscore t1 were proven to be
beneficial for classification, scoring much a higher importance. 

The above mentioned results demonstrate that the methodology works well.
Nevertheless, the final scores are similar to already published results.
This highlights the need to explore new directions to construct better
classifiers which are usable in next-generation astronomical surveys.   
As usual, a critical improvement can come from better observational
data.
An improved transient recognition could be achieved by training on
data with lower  amount of duplicates and errors, i.e. a more reliable
non-transient data-set.    
Expanding the methodology presented may be beneficial too. 
Using additional features could also increase the classifiers
performance. 
Moreover, different classification algorithms can also be tested for a
better detection of transient objects.  
Finally, testing with other more reliable oversampling types would be
beneficial for the purpose of working with limited data-sets. 
While the oversampled/balanced inputs increased the performance of some of
the algorithms, different alternatives could improve upon the results
shown in this document. 



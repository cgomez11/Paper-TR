% mnras_template.tex
%
% LaTeX template for creating an MNRAS paper
%
% v3.0 released 14 May 2015
% (version numbers match those of mnras.cls)
%
% Copyright (C) Royal Astronomical Society 2015
% Authors:
% Keith T. Smith (Royal Astronomical Society)

% Change log
%
% v3.0 May 2015
%    Renamed to match the new package name
%    Version number matches mnras.cls
%    A few minor tweaks to wording
% v1.0 September 2013
%    Beta testing only - never publicly released
%    First version: a simple (ish) template for creating an MNRAS paper

%%%%%%%%%%%%%%%%%%%%%%%%%%%%%%%%%%%%%%%%%%%%%%%%%%
% Basic setup. Most papers should leave these options alone.
\documentclass[a4paper,fleqn,usenatbib]{mnras}

% MNRAS is set in Times font. If you don't have this installed (most LaTeX
% installations will be fine) or prefer the old Computer Modern fonts, comment
% out the following line
%\usepackage{newtxtext,newtxmath}
\usepackage{import}
% Depending on your LaTeX fonts installation, you might get better results with one of these:
%\usepackage{mathptmx}
%\usepackage{txfonts}

% Use vector fonts, so it zooms properly in on-screen viewing software
% Don't change these lines unless you know what you are doing
\usepackage[T1]{fontenc}
\usepackage{ae,aecompl}



%%%%% AUTHORS - PLACE YOUR OWN PACKAGES HERE %%%%%

% Only include extra packages if you really need them. Common packages are:
\usepackage{graphicx}	% Including figure files
\usepackage{amsmath}	% Advanced maths commands
\usepackage{amssymb}	% Extra maths symbols

% \usepackage{multicol}        % Multi-column entries in tables
% \usepackage{bm}		% Bold maths symbols, including upright Greek
\usepackage{pdflscape}	% Landscape pages
\usepackage[utf8]{inputenc}
\usepackage{multirow}
\usepackage{float} % here for H placement parameter

%%%%%%%%%%%%%%%%%%%%%%%%%%%%%%%%%%%%%%%%%%%%%%%%%%

%%%%% AUTHORS - PLACE YOUR OWN COMMANDS HERE %%%%%

% Please keep new commands to a minimum, and use \newcommand not \def to avoid
% overwriting existing commands. Example:
%\newcommand{\pcm}{\,cm$^{-2}$}	% per cm-squared

%%%%%%%%%%%%%%%%%%%%%%%%%%%%%%%%%%%%%%%%%%%%%%%%%%

%%%%%%%%%%%%%%%%%%% TITLE PAGE %%%%%%%%%%%%%%%%%%%

% Title of the paper, and the short title which is used in the headers.
% Keep the title short and informative.
\title[Transient Event Recognition using Machine Learning]{Transient Event Recognition using Machine Learning}

% The list of authors, and the short list which is used in the headers.
% If you need two or more lines of authors, add an extra line using \newauthor
\author[Diego. A. Gomez et al.]{
Diego A. Gomez,$^{1}$\thanks{E-mail: da.gomez11@uniandes.edu.co}
Marcela Hernandez$^{1}$   
Jaime Forero-Romero$^{2}$
and Pablo Arbelaez$^{3}$
\\
% List of institutions
$^{1}$Departamento de Ingenier\'ia de Sistemas y Computaci\'on, Universidad de los Andes, Cra. 1 No. 18A-10, Bogot\'a, Colombia\\
$^{2}$Departamento de F\'isica, Universidad de los Andes, Cra. 1 No. 18A-10, Bogot\'a, Colombia\\
$^{3}$Departamento de Ingenier\'ia Biom\'edica, Universidad de los Andes, Cra. 1 No. 18A-10, Bogot\'a, Colombia
}

% These dates will be filled out by the publisher
\date{Accepted XXX. Received YYY; in original form ZZZ}

% Enter the current year, for the copyright statements etc.
\pubyear{2018}

% Don't change these lines
\begin{document}
\label{firstpage}
\pagerange{\pageref{firstpage}--\pageref{lastpage}}
\maketitle

% Abstract of the paper
\begin{abstract}

%\import{./chapters/}{0.abs.tex}
% This is a simple template for authors to write new MNRAS papers.
% The abstract should briefly describe the aims, methods, and main results of the paper.
% It should be a single paragraph not more than 250 words (200 words for Letters).
% No references should appear in the abstract.



The arrival of massive multi-epoch and multi-band astronomical surveys
demands the development of computational techniques to automate the
study and detection of transient astronomical sources. 
In this work we characterize various machine learning algorithms to
recognize and classify such events.
We use light-curves from the Catalina Real Time Transient Survey
(CRTS) to classifity thousands of unique transient and non-transient
event light-curves.
We experiment  with multiple data pre-processing methods,
feature selection techniques and self-learning models. 
We obtain new results in five classification tasks, paying particular
attention to a binary and a four multi-class classifications task.  
We conclude that the best performing algorithm was
a Random Forest with an f1-score of 87.27\% on binary
(transient \& non-transient) classification. 
Six-class transient classification achieved a 77.54\% f1-score, and a
66.39\% f1-score when including an ambiguous sources class.
Multi-class transient classifications including non-transient sources,
resulted in a 75.05\% f1-score and a 66.05\% f1-score respectively.  

\end{abstract}

% Select between one and six entries from the list of approved keywords.
% Don't make up new ones.
\begin{keywords}
methods: data analysis, statistical
%keyword1 -- keyword2 -- keyword3
\end{keywords}

%%%%%%%%%%%%%%%%%%%%%%%%%%%%%%%%%%%%%%%%%%%%%%%%%%

%%%%%%%%%%%%%%%%% BODY OF PAPER %%%%%%%%%%%%%%%%%%

\section{Introduction}
%\import{./chapters/}{1.intro.tex}
% All papers should start with an Introduction section, which sets the work
% in context, cites relevant earlier studies in the field by \citet{Others2013},
% and describes the problem the authors aim to solve \citep[e.g.][]{Author2012}.

%%% NEW MULTI EPOCH SURVEYS ENABLE STUDY AND DETECTION IN CRAZY SCALES
The study and detection of astronomical variable sources is expected
to occur on unprecedented scales with the new generation of
forthcoming multi-epoch and multi-band (synoptic) astronomical
surveys. 
For instance, the Large Synoptic Survey Telescope (LSST)
\citep{0805.2366}, one of the largest synoptic survey telescopes to
come in the following years, will generate approximately 15 terabytes
of data every night \citep{1512.07914}.  
%Such telescope is expected to detect and alert about 10 million possible transients too. 
Other surveys including the Square Kilometer Array (SKA) are also expected
to generate exuberant daily data-streams. 

%%% HISTORICALLY IT HAS BEEN DONE MANUALLY, NOW IT ISN"T POSSIBLE
This observational leap renders manual classification techniques
unfeasible.   
Traditionally, such objects have been classified by visual inspection
by experts or through crowd-sourcing
\citep{1011.2199,0708.2750}. 
This is approach is slow and expensive.
Another concern is the possible biases and difficulty to standardize among 
astronomers \citep{1104.3142}. 
Alternatively, transient detection can be executed much faster than
human astronomers through computational techniques using machine
learning, 
which are deterministic and calculate the results' degrees of
certainty. 
These methods also allow for real-time triggering of follow-up
observations that optimize the economical and temporal resources. 
 

%%% TRANSIENTS EVENTS ARE CHALLENGES (AND WHAT TRANSIENT EVENTS ARE)
Another challenge in  Time Domain Astronomy is Real-Time Transient classification. 
Astronomical Transients are events which's luminosity varies in short duration
relative in the timescale of the universe, from minutes to several
years. 
Transients include phenomena such as supernovae, novae, neutron
stars, blazars, pulsars, cataclysmic variable stars (CV), gamma ray
bursts (GRB) and active galaxy nucleus (AGN). 
The time-domain dependency of these objects is one of the reasons why
they are hard to classify: their data is usually heterogeneous,
unbalanced, sparse, unevenly sampled and with missing information. 
Automating the recognition and classification of transient events, a type of such
variable sources, would reduce costs and speed up this process as well
as provide scientists information of the universe in various spatial
scales \citep{2011arXiv1110.4655D}.  

%%% HOW DOES TRANSIENT DETECTION
Transient detection is generally done through 
difference imaging \citep{1507.05137,1608.01733,1708.02850}. 
This algorithm starts by aligning the image of interest on a reference image of the
same region of sky, processing the former to match its point-spread
function in all regions with the latter's, and subtracting both
\citep{astro-ph/9712287}. 
As a result, variable stars and transient objects which were not
visible in the reference image will remain in the resulting
image. 
Nonetheless, the difference images can also contain bogus
artifacts due to imperfections in the image processing
phase, in the telescope or other phenomena. 
Unfortunately, distinguishing between bogus and real transient objects
still requires human intervetion, making it expensive and slow
\citep{2011arXiv1110.4655D}.  


%%% STATE OF THE ART I: REAL VS BOGUS
There are succesful attempts to implement automatic detection
algorithms and distinguish bogus artifacts from real transient
objects.  
For instance, raw images from the  Skymapper Supernova and Transient
Survey and the High cadence Transient Survey (HiTS) have been used as
inputs for automatic detection algorithms \citep{1708.08947,1701.00458}.
Convolutional Neural Networks (CNN) have also achieved
high accuracy in this binary classification task.
Other studies have shown that bogus artifacts can be detected using
features extracted from raw images. 
\cite{1601.06151} and \cite{1601.06320}, achieved reliable
classification by transforming transient data from the OGLE-IV
data-reduction pipeline and training it with machine learning
algorithms such as Artificial Neural Networks, Support Vector
Machines, Random Forests, Naive Bayes, K-Nearest Neighbors and Linear
Discriminant Analysis.  
Similarly, \cite{1501.05470} used images from Pan-STARS1 Medium Deep
Surveys, and \cite{1407.4118} processed single-epoch multi-band images
from the SDSS supernova survey for the same purpose.  


%%% STATE OF THE ART II: DIRECT

Some alternative approaches classify astronomical transient
events skipping the difference imaging process. 
This is usually done by extracting relevant features (e.g. periodic,
non-periodic) from the astronomical object light-curves, and using
machine learning algorithms for classification. 
For instance, in \cite{1401.3211} and \cite{1601.03931}, light curves
of objects six or more transient classes from the Catalina Real Time
Transient Survey and the Downes set (\cite{d05}) were
classified using this approach. 
Other studies which research a similar approach
\cite{1603.00882} used the same approach to finf 
where supernovas from the Supernova Photometric Classification
Challenge.
Recurrent Neural Networks (RNNs) have also been proven succesful to
classify transient events, \cite{1606.07442} and \cite{1710.06804}
showed that  modern algorithms can also learn from time-series data
without expensive image processing.    

% PROPOSAL
In this paper we follow this approach and test different machine
learning algorithms to classify transient events using light curves as
an input.  
We start by extracting several statistical descriptors from the light
curves.
These descriptors are the input of three different machine learning
models with several hyper-parameter variations: Support Vector
Machines, Neural Networks and Random Forests, which automatically
learn to detect and classify between different types of transient
objects. 
We present the results of extensive testing and multiple experiments 
executed in order to find the best training parameters. 

% STRUCTURE
The paper is structured as it follows. In Section \ref{section_data}
we present the dataset used for this project. 
Section \ref{section_method} describes the methodology.
Section \ref{section_experimentation} explains the experiments
performed to classify transient objects with mach ine
learning. 
Finally, the results are presented and discussed in
Section \ref{section_results}.  

\section{Data} \label{section_data}
%\import{./chapters/}{2.data.tex}
%%% Overview

We use public data from the Catalina Real-Time Transient Survey
(CRTS) \citep{1111.2566}, an astronomical survey searching transient
and highly variable objects.   
It covered 33000 squared degrees of sky and took data since 2007.
Three telescopes were used: Mt. Lemmon Survey (MLS), Catalina Sky
Survey (CSS), and Siding Spring Survey (SSS). So far, CRTS has
discovered more than $15000$ transient events.
We use light curves as measured with the CSS telescope of the CRTS, which is
an f/1.8 Schmidt telescope located in the Santa Catalina Mountains, north of Tucson,
Arizona and is equipped with a 111-megapixel  detector, and covered
4000 square degrees per night, with a limiting magnitude of 19.5 in
the V band.  
The public CRTS data base reports the source flux and its
corresponding uncertainty \citep{1996PASP..108..851S}.


All transient objects were classified in the CRTS data-set according to
their type. 
The most relevant classes found are: supernovae (SN),
cataclysmic variable stars (CV), blazars, flares, asteroids, active
galactic nuclei (AGN), and high-proper-motion stars (HPM). 
Though most objects in the transient object catalogue belong to a single class,
there is some uncertainty in the categorization of some of
them. 
In this case, an interrogation sign is used when a class is not clear
e.g. SN? or sometimes multiple possible classes are found for a single
event e.g. SN/CV)
Table \ref{Top-Transient-Classes} summarized The number of objects in each class.


We use the light curves of $4269$ unique transient events that
contain at least 5 observations.
We also use  $15193$ non-transient sources with at least 5 observations each. 
Sources in this data-set were selected by sampling light curves
of objects within a 0.006 degree radius from CRTS detected transients,
and removing known transient light curves from that set. 
Though this process should return only non-transient sources, it is
possible that non-detected transients were captured and catalogued
incorrectly as 'non-transients'.  
Table \ref{Transient-Observation-Count} and
Table \ref{Non-Transient-Observation-Count} summarize some statitics
on the number of observations available for each light curve for the
transient and non-transient data-sets, respectively.  



\import{./figures/ObjCount/}{5.tex}


\section{Methodology} \label{section_method}
%\import{./chapters/}{3.method.tex}

Our methodology has six stages: 
filtering out irrelevant light curves, oversampling light curves, extracting
descriptive features from the curves, processing them into feature
vectors, re-scaling those resulting feature vectors, and 
classification with machine learning algorithms. 
The details of each step are in the following subsections.

\subsection{Data Filtering} \label{subsection_filtering}

We discard light curves with few observations as they may not contain
enough information to be classified correctly by machine learning
algorithms.  
The nominal cut is 5 observations per light curve, but we also test a
higher cut by filterig out light curves with less than 10 observations.

\subsection{Oversampling Transient Light Curves} \label{subsection_oversampling}

The number light curves per class is unbalanced. 
In order to have the same amount elements for each class we implement an
oversampling step by artificially generating multiple mock light curves,
each based on an observed one. 

We generate a mock light curve from the observed light curve and 
then sample the observed magnitude from a Gaussian distribution
centered on the observational apparent magnitude with the magnitude's
error as the  standard deviation. 

{\bf Y entonces se generaron al final 1293 curvas para todos las
  clases para tener el mismo numero de elementos que la clase SN?}

\subsection{Feature Extraction} \label{subsection_extraction}

Light curves are sampled at irregular time intervals and have
different numbers of data points.
This makes it challenging to directly use the time-series data for
classification with traditional methods. 
We circumvent this problem by extracting a set of features for each
light curve.
These features are scalars dereive from statistical and model-specific
fitting techniques.
Some of these features were  were formally introduced in
\cite{1101.1959}, and have been used in other studies
\citep{1603.00882,1601.03931}.  

We use 30 features in total which can be classified in four groups:
moment-based, magnitude-based, percentile-based and fitting-based. 
These features are.

\begin{enumerate}
    
\item Moment-based features use the magnitude for each light curve.
  \begin{itemize}
  \item \underline{Beyond1std} (\textit{beyond1std}): 
    Percentage of observations which are over or under one standard
    deviation from the weighted average. Each weight is calculated as
    the inverse of the corresponding observation's photometric error. 
        \item \underline{Kurtosis} (\textit{kurtosis}): 
        The fourth moment of the data distribution. Used to measure
        the heaviness or lightness in the tails of the statistical
        data. 
        \item \underline{Skewness} (\textit{skew}): 
        A measurement of the level of asymmetry from the normal
        distribution in a data distribution. Negative skewness is the
        property of a more pronounced left tail, while positive
        skewness is a characteristic that implies a more pronounced
        right tail. 
        \item \underline{Small Kurtosis} (\textit{sk}):
        Small sample kurtosis.
        \item \underline{Standard deviation} (\textit{std}):
        The standard deviation of the magnitudes..
        \item \underline{Stetson J} (\textit{stetson\textunderscore j}):
        The Welch-Stetson J variability index
        \cite{1996PASP..108..851S}. A robust standard deviation. 
        \item \underline{Stetson K} (\textit{stetson\textunderscore k}): 
        The Welch-Stetson K variability index
        \cite{1996PASP..108..851S}. A robust kurtosis measure. 
    \end{itemize}
    
    \item Magnitude-based use the magnitude for each source.
    \begin{itemize}
    \item \underline{Amplitude} (\textit{amp}): 
      The difference between the maximum and minimum magnitudes.
    \item \underline{Max Slope} (\textit{max\textunderscore slope}): 
      Maximum absolute slope between two consecutive observations.
    \item \underline{Median Absolute Deviation} (\textit{mad}): 
      The median of the difference between magnitudes and the median
      magnitude. 
    \item \underline{Median Buffer Range Percentage} (\textit{mbrp}): 
      The percentage of points within 10\% of the median magnitude.
    \item \underline{Pair Slope Trend} (\textit{pst}): 
      Percentage of all pairs of consecutive magnitude measurements that have positive slope.
    \item \underline{Pair Slope Trend 30} (\textit{pst\textunderscore last30}): 
      Percentage of the last 30 pairs of consecutive magnitudes that have a positive slope, minus percentage of the last 30 pairs of consecutive magnitudes with a negative slope.
    \end{itemize}


  \item Percentile-based features use the sorted flux distribution for
    each source. The flux is computed as $F = 10^{0.4 \mathrm{mag}}$. 
    We define $F_{n,m}$ as the difference between the $m$-th and $n$-the flux
    percentiles. 
    \begin{itemize}
    \item \underline{Percent Amplitude} (\textit{p \textunderscore amp}): 
      Largest percentage difference between the absolute maximum magnitude and the median.
    \item \underline{Percent Difference Flux Percentile} (\textit{pdfp}): 
      Ratio between $F_{5,95}$ and the median flux.
    \item \underline{Flux Percentile Ratio Mid20} (\textit{fpr20}): 
      Ratio $F_{40,60} / F_{5,95}$
    \item \underline{Flux Percentile Ratio Mid35} (\textit{fpr35}):
        Ratio $F_{32.5,67.5} / F_{5,95}$
      \item \underline{Flux Percentile Ratio Mid50} (\textit{fpr50}): 
        Ratio $F_{25,75} / F_{5,95}$
      \item \underline{Flux Percentile Ratio Mid65} (\textit{fpr65}): 
        Ratio $F_{17.5,82.5} / F_{5,95}$
      \item \underline{Flux Percentile Ratio Mid80} (\textit{fpr80}): 
        Ratio $F_{10,90} / F_{5,95}$
    \end{itemize}
    
  \item Polynomial Fitting-based features are the coefficients of
    multi-level terms in polynomial curve fitting. This is new set
    of features proposed in this paper. 
    \begin{itemize}
        \item \underline{Poly1 T1}: Linear term coeff. in monomial curve fitting.
        \item \underline{Poly2 T1}: Linear term coeff. in quadratic curve fitting.
        \item \underline{Poly2 T2}: Quadratic term coeff. in quadratic curve fitting.
        \item \underline{Poly3 T1}: Linear term coeff. in cubic curve fitting.
        \item \underline{Poly3 T2}: Quadratic term coeff. in cubic curve fitting.
        \item \underline{Poly3 T3}: Cubic term coeff. in cubic curve fitting.
        \item \underline{Poly4 T1}: Linear term coeff. in quartic curve fitting.
        \item \underline{Poly4 T2}: Quadratic term coeff. in quartic curve fitting.
        \item \underline{Poly4 T3}: Cubic term coeff. in quartic curve fitting.
        \item \underline{Poly4 T4}: Quartic term coeff. in quartic curve fitting.
    \end{itemize}    
\end{enumerate}

We group these features into three sets for experimentation:

\begin{enumerate}
\item \underline{20 feats:} Includes all moment-based, magnitude-based
  and percentile-based features.  
\item \underline{26 feats:} Includes all features, except quartic
  curve fitting parameters (poly4). 
\item \underline{30 feats:} Includes all features.
\end{enumerate}


\subsection{Feature Scaling} \label{subsection_scaling}

We use two feature scaling procedures to weight equally each input
in the machine learning training process.
These procedures are applied to the magnitudes for each light curve.
The first is standardization, where the re-scaled values have zero mean and unit variance. 
The second is normalization, were the magnitudes are re-scaled  to the
range $[0,1]$. 
These procedures are part of the modules that are switched on/off as
part of the experimentation explained in Section \ref{section_experimentation}.


\subsection{Classification} \label{subsection_classification}

We perform five classification tasks.

\begin{enumerate}
\item \textbf{Binary Classification}: 
  Distinguish transients from non-transients. A balanced number of
  events from both classes is used to investigate the capability of
  distinguishing between these different types. 
\item \textbf{6-Transient Classification}: Recognize objects as
  belonging to one of the more common transient types in the dataset,
  namely: AGN, Blazar, CV, Flare, HPM and Supernovae. The purpose of
  this task was to test how well does classification algorithms perform
  when distinguishing only the main transient event types in the
  dataset. 
\item \textbf{7-Transient Classification}: Recognize objects as
  belonging to one of the more common transient types in the dataset
  or as a different transient, namely: AGN, Blazar, CV, Flare, HPM,
  Other and Supernovae. The Other class was created by using objects
  from ambiguous and under-represented classes such as the ones
  described in Section \ref{section_data}. In this scenario we wanted
  to understand how well these Other classes are detected when
  grouped together, and how the performance of the 6-Transient
  Classification Task changes with the addition this new class. 
    \item \textbf{7-Class Classification}: Recognize objects as
      belonging to one of the more common transient types in the
      dataset or as a Non-Transient object. We use the classes:
      AGN, Blazar, CV, Flare, HPM, Non-Transient and Supernovae. 
      The purpose is understanding whether classifiers could classify
      correctly the main transient types when the non-transient events
      are present.  
    \item \textbf{8-Class Classification}: Recognize objects as
      belonging to any of the the following classes: AGN, Blazar, CV,
      Flare, HPM, Other, Non-Transient and Supernovae. 
      The Other classs is built using objects from ambiguous and
      under-represented classes. 
      This task quantifies how good is the classification when testing
      together all the main classes in the previous tasks.
\end{enumerate}

We use three classification algorithms: Neural Networks, Random
Forests and Support Vector Machines. 
These algorithms are popular in published studies and are efficient 
for low dimensional feature datasets as is our case. 
Details on the inner workings of these machine learning models can be
found in \cite{9780387848570}.  

We also make sure that the training data does not contain oversampled
light curves based on any other present in the test samples, since
this could introduce a bias.  
The test samples do not contain not contain oversampled light
curves, for the same reason. 
Additionally, since the number of objects in the class sub-sets was
small we used 2-fold cross validation during training for result
validation. 
Moreover, grid search was used during training to test multiple
hyper-parameter configurations for each one  of the possible
algorithms. 
We use the F1-Score to assess the performance of a given model.
We initially define a test set aside for each task and compute 
the final scores based on the prediction performance of the trained
algorithms with these data subsets.  


\subsection{Feature Importance} \label{subsection_importances}

After finding the best classifiers we generate  a list of the most
relevant features. 
We do this with  the best Random Forest classifier for each task. 
This approach quantifies  which features contributed the most to the
classification task.  

\section{Experiments} \label{section_experimentation}

%\import{./chapters/}{4.exp.tex}
We test all classification tasks described earlier with multiple
parameter variations.  
We run multiple times each task with different criteria 
(dataset inputs, pre-processing steps and algorithms), 
to find out  what configuration worked the best for each case. 

We use five different parameters in our experiments.
Each expriment includes at least two possible values for each
parameter.
We try all possible combinations of all parameter variations for the
five classification tasks. 
These five parameters are the following. 

\begin{description}
    \item \textbf{Minimum Observations} 
      This quantifies how       different amounts of observations
      per light curve impact  transient classification.
      We use two values for this parameter: 5 and 10 (Section \ref{subsection_filtering}). 
    \item \textbf{Balancing Classes} 
      The performance of classification algorithms performance may differ if the number
      of elements for each class is not equal during training. 
      We experiment with two different scenarios.
      scenarios. 
      First, we use the original CRTS light curves, with unbalanced transient classes (Table
      \ref{Top-Transient-Classes}).
      Second, we use a sub-set of light curves with balanced classes. 
      To obtain balanced classes, those with a smaller count than the
      biggest class are incremented with synthetic light curves
      generated as  described Section \ref{subsection_oversampling}.              
      Note that the Binary Classification task described in Section
      \ref{section_method} has by construction balanced classes only.
    
    \item \textbf{Number of Features} In order to test how much our classification results varied when using different feature subsets, three different subsets were tested for each task: 20, 26 or 30 features as defined in Section \ref{subsection_extraction}.
    \item \textbf{Feature Scaling} As described in Section \ref{subsection_scaling}, we tested independently scaling the features using either min-max scaling or standardization, since different feature scaling methods may result in distinct results.
    \item \textbf{Model Used} Three substantially different algorithms were tested, in order to investigate which learned the better from the data. Namely, they were Neural Networks, Random Forests or Support Vector Machines (Section \ref{subsection_classification}).
\end{description}



\section{Results} \label{section_results}
\import{./chapters/}{5.res.tex} 

\section{Conclusions}
%\import{./chapters/}{6.conc.tex}

The scope of forthcoming of large astronomical synaptic surveys such
as the LSST \citep{0805.2366} motivates the development and
exploration of authomatized ways to detect transient sources.
In this paper we presented an approach for the automatic recognition
of transient events with machine learning techniques.   

The method we present is based on the study of light curves. 
We extracted its characteristic features to use them as inputs
to train three different machine learning algorithms: Random Forests,
Neural Networks and Support Vector Machines.
The features extracted from light curves were either statistical
descriptors of the observ ations, or polynomial curve fitting
coefficients applied to the light curves.   

The machine learning algorithms performed a variety of classification tasks.
Namely, binary classification among transients and non-transients, and
multi-class classification of various transient classes, sometimes
including non-transients too. Detailed description of these tasks can
be found in Section \ref{section_method}.  
Overall, the best classifier for all tasks was the Random Forest,
followed by Neural Networks and then Support Vector Machines. 

State of the art results were obtained when testing the trained models
to unseen sets of data, specifically in binary and transient
multi-class classification. 
Recall scores obtained from the best classifier for each task, are
comparable to those results found in \cite{1401.3211}
and \cite{1601.03931}.   


We also investigated the parameters that produced the best results.
Training with light curves which contained 10 observations minimum
generally outperformed those with minimum 5. 
Overall, the best results were also achieved using all 30 features per
light curve, while using 20 features performed worst. 
Such phenomena may be explained due to the lower noise that curves
with more observations provide.  
Since light-curves with higher observation counts may be better
defined, they could generate a more precise higher rank polynomial
fitting than non-transients, which were the newly proposed features
found in the 26 and 30 feature sets. 

In general, training with unbalanced-class data sets resulted in the
highest values. 
Contrary to what was expected, the usage of oversampled light-curves
may be a factor that biases the model during training. 
This could be explained if such artificial light curves were too
similar to the original ones, and thus the algorithms over-fitted
during training.  

We studied feature importance using Random Forests. 
The most importante feature was always stetson\textunderscore j, followed by:
amplitude, standard deviation, skewness and median absolute
deviation. 
Conversely, 4th level polynomial fitting coefficients were
the ones which provided the least relevance, though using them was
still better, since the highest scores were a result of using all 30
features. 
Furthermore, lower polynomial fitting features like
poly2\textunderscore t1 and poly1\textunderscore t1 were proven to be
beneficial for classification, scoring much a higher importance. 

The above mentioned results demonstrate that the methodology works well.
Nevertheless, the final scores are similar to already published results.
This highlights the need to explore new directions to construct better
classifiers which are usable in next-generation astronomical surveys.   
As usual, a critical improvement can come from better observational
data.
An improved transient recognition could be achieved by training on
data with lower  amount of duplicates and errors, i.e. a more reliable
non-transient data-set.    
Expanding the methodology presented may be beneficial too. 
Using additional features could also increase the classifiers
performance. 
Moreover, different classification algorithms can also be tested for a
better detection of transient objects.  
Finally, testing with other more reliable oversampling types would be
beneficial for the purpose of working with limited data-sets. 
While the oversampled/balanced inputs increased the performance of some of
the algorithms, different alternatives could improve upon the results
shown in this document. 



% \section*{Acknowledgements}

% The Acknowledgements section is not numbered. Here you can thank helpful
% colleagues, acknowledge funding agencies, telescopes and facilities used etc.
% Try to keep it short.

%%%%%%%%%%%%%%%%%%%%%%%%%%%%%%%%%%%%%%%%%%%%%%%%%%

%%%%%%%%%%%%%%%%%%%% REFERENCES %%%%%%%%%%%%%%%%%%

% The best way to enter references is to use BibTeX:

\bibliographystyle{mnras}
\bibliography{bibliography}
%\bibliography{example} % if your bibtex file is called example.bib


% Alternatively you could enter them by hand, like this:
% This method is tedious and prone to error if you have lots of references
% \begin{thebibliography}{99}
% \bibitem[\protect\citeauthoryear{Author}{2012}]{Author2012}
% Author A.~N., 2013, Journal of Improbable Astronomy, 1, 1
% \bibitem[\protect\citeauthoryear{Others}{2013}]{Others2013}
% Others S., 2012, Journal of Interesting Stuff, 17, 198
% \end{thebibliography}

%%%%%%%%%%%%%%%%%%%%%%%%%%%%%%%%%%%%%%%%%%%%%%%%%%

%%%%%%%%%%%%%%%%% APPENDICES %%%%%%%%%%%%%%%%%%%%%

\appendix
\import{./chapters/}{7.appendix.tex}
% \section{Some extra material}

% If you want to present additional material which would interrupt the flow of the main paper,
% it can be placed in an Appendix which appears after the list of references.

%%%%%%%%%%%%%%%%%%%%%%%%%%%%%%%%%%%%%%%%%%%%%%%%%%


% Don't change these lines
% \bsp	% typesetting comment
\label{lastpage}
\end{document}

% End of mnras_template.tex
